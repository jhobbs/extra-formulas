\documentclass[10pt]{article}
\usepackage{siunitx}
\usepackage{commath}
\begin{document}
\section*{Complex Numbers}
To convert from rectangular ($a + jb$) to polar form ($r \angle \theta$):\\\\
$r = \sqrt{(a^2 + b^2)}$\\
$\theta = \tan^{-1}(\frac{a}{b})$\\\\
To convert from polar form to rectangular form:\\\\
$a = r\cos\theta$\\
$b = r\sin\theta$\\\\
The reciprocal of $j$ is $-j$:\\\\
$1/j = -j$
\section*{RC Circuit Time Constants}
Time constant for a simple RC circuit:\\\\
$\tau = RC$\\\\
where $R$ is resistance in ohms, and $C$ is capacitance in farads.\\\\
Voltage of a charging capacitor at time $t$:\\\\
$V(t) = E(1 - e^{-t/\tau})$\\\\
where $V(t)$ is the voltage across the capacitor at time $t$, $E$ is the applied voltage, and $\tau$ is the time constant of the circuit.\\\\
Voltage of a discharging capacitor at time $t$:\\\\
$V(t) = E(e^{-t/\tau})$\\\\
where $V(t)$ is the voltage across the capacitor at time $t$, $E$ is the applied voltage, and $\tau$ is the time constant of the circuit.\\\\
For a charging capacitor, $V(\tau) = 0.632E$ and $V(5\tau) = 0.993E$.\\
For a discharging capacitor, $V(\tau) = 0.368E$ and $V(5\tau) = 0.007E$.
\section*{RL Circuit Time Constants}
Time constant for a simple RL circuit:\\\\
$\tau = \frac{L}{R}$\\\\
where $R$ is resistance in ohms, and $L$ is inductance in henries.\\\\
Current of a charging capacitor at time $t$:\\\\
$I(t) = \frac{E}{R}(1-e^{-t/\tau})$\\\\
For a an inductor ramping to 100A, $I(\tau) = 63.2$A and $I(5\tau) = 99.3$A.
\section*{Computing Impedance}
Inductant Reactance:\\\\
$X_L = 2\pi fL$\\\\
where $X_L$ is inductive reactance in ohms, f is frequency in hertz, and L is inductance in henries.\\\\
Capacitive Reactance:\\\\
$X_C = \frac{1}{2\pi fC}$\\\\
where $X_C$ is capacitave reactance in ohms, f is frequency in hertz, and C is capacitance in farads.\\\\
Reactance:\\\\
$X = X_L - X_C$\\\\
Admittance is the reciprocal of impedance:\\\\
$Y = 1/Z$\\\\
Conductivity is the reciprocal of resistance:\\\\
$G = 1/R$\\\\
Suspectance is the reciprocal of reactance:\\\\
$B = 1/X$\\\\
Impedances in series add together.\\
Admittances in parallel add together.\\\\
Impedance magnitude:\\\\
$\abs{Z}=\sqrt{R^2 + X^2}$\\\\
Phase angle:\\\\
$\theta = \tan^{-1}(\frac{R}{X})$\\\\
Rectangular Impedance:\\\\
$Z = R + jX$\\\\
where X is positive for inductive reactance and negative for capacitive reactance.\\\\
Polar Impedance:\\\\
$Z = \abs{Z}\angle\theta$
\section*{Reactive Power}
Apparent Power:\\\\
$P_{APPARENT} = IE$\\\\
For a series circuit:\\\\
$P_{REAL} = I^2R$\\\\
where $I$ is the RMS current.\\\\
For a parallel circuit:\\\\
$P_{REAL} = \frac{E^2}{R}$\\\\
where $E$ is the RMS voltage.\\\\
Power factor:\\\\
$PF = \frac{P_{REAL}}{P_{APPARENT}}$\\
$PF = \cos\theta$\\\\
where $\theta$ is the phase angle.
\section*{Resonant Circuits}
Resonant frequency of an LC circuit:\\\\
$f_r = \frac{1}{2\pi\sqrt{LC}}$\\\\
Solve for inductance with given frequency and capacitance:\\\\
$L=\frac{1}{(2\pi f_r)^2C}$\\\\
Solve for capacitance with given frequency and inductance:\\\\
$C=\frac{1}{(2\pi f_r)^2L}$\\\\
Quality Factor:\\\\
$Q = \frac{X}{R}$\\\\
Half Power (-3db) Bandwidth:\\\\
$\Delta f = \frac{f_r}{Q}$


\section*{Computing Inductance}
Inductance for N wraps on a Powdered Iron core:\\\\
$L=\frac{A_L N^2}{10,000}$\SI{}{\micro\henry}\\\\
Number of wraps to acheive L \SI{}{\micro\henry} of inductance on a Powdered Iron core:\\\\
$N=100\sqrt{\frac{L}{A_L}}$ wraps\\\\
Inductance for N wraps on a Ferrite core:\\\\
$L=\frac{A_L N^2}{1,000,000}$\SI{}{\milli\henry}\\\\
Number of wraps to acheive L \SI{}{\milli\henry} of inductance on a Ferrite core:\\\\
$N=1000\sqrt{\frac{L}{A_L}}$ wraps
\end{document}
